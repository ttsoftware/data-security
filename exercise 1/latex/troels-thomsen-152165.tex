\documentclass[12pt]{article}
\usepackage[a4paper, hmargin={2.5cm, 2.5cm}, vmargin={2.5cm, 2.5cm}]{geometry}

\usepackage[utf8]{inputenc}
\usepackage[english]{babel}
\usepackage{amssymb}
\usepackage{amsfonts}
\usepackage{amsmath}
\usepackage{setspace}
\usepackage{algorithm}
\usepackage[noend]{algpseudocode}

\usepackage{tikz}
\usetikzlibrary{positioning,shapes, shadows, arrows, automata}

\usepackage{xcolor}
\usepackage{listings}
\usepackage{graphicx}
\usepackage[hidelinks]{hyperref}
\usepackage{float}
\usepackage[english]{varioref}
\usepackage{multirow}
\usepackage{hhline}
\usepackage{etoolbox}

\usepackage{fancyhdr}

\setlength\parindent{0pt}
\usepackage[parfill]{parskip}

\definecolor{mygray}{rgb}{0.9451,0.9451,0.9451}
\lstset{
  backgroundcolor=\color{mygray},
  basicstyle=\footnotesize\ttfamily,
  mathescape,
  breaklines=true,
  numbers=left,
  numberstyle=\ttfamily,
  stepnumber=1,
  firstnumber=1,
  numberfirstline=true,
  postbreak=\raisebox{0ex}[0ex][0ex]{\ensuremath{\color{red}\hookrightarrow\space}},
  literate={->}{$\rightarrow$}{2}
           {ε}{$\varepsilon$}{1}
}

\linespread{1.3}

\title{
  \vspace{4cm}
  \begin{flushleft}
  \Large{\textbf{Excercise 1}} \\
  \large{Data Security}
  \end{flushleft}
  \vspace{0cm}
  \begin{flushleft}
  \small
  \textit{\today}
  \end{flushleft}
  \vspace{12cm}
  \begin{flushleft}
  \small
  Troels Thomsen \texttt{152165}
  \end{flushleft}
}

\date{
	%
}

\begin{document}

\clearpage
\pagenumbering{gobble}
\thispagestyle{empty}
\maketitle

\newpage

\pagenumbering{arabic}

\section{Kerberos PKInit}

\subsection{Protocol description}
\label{sub:Protocol description}

The protocol describes how $C$ authenticates with- and receives payload from the server $s$.

$C$ establishes a secure connection with $g$, by obtaining the symmetric keys $KCG$ and $Ktemp$ from $ath$. These two keys are only shared between $ath$ and $C$. $ath$ also sends the secret $\{|ath,C,g,KCG,T1|\}skag$ to $C$, which $C$ will use to verify that it got $KCG$ from $ath$. $C$ cannot read or modify this secret since it was encrypted using $skag$, which is only shared between $ath$ and $g$.

$C$ can now talk to $g$, saying it wants to contact $s$, sending the $skag$-encrypted secret containing $KCG$ and $\{|C,T1|\}KCG$ which is used to verify $C$ and the nons $T1$ from $ath$.

If the information supplied by $C$ is correct, $g$ responds with the secret $\{|C,s,KCS,T2|\}skgs$ which only $g$ and $s$ can read since it is encrypted using their shared symmetric key $skgs$.
$g$ also sends $\{|s,KCS,T2,N2|\}KCG$, giving $C$ the symmetric key $KCS$ which is only shared between $C$ and $s$.

$C$ can now contact $s$ by sending $KCS$ encrypted with $skgs$ and it's own identifier together with $g$'s $T2$ nons encrypted using $KCS$.

$s$ responds with the payload encrypted with $KCS$.

The protocol prevents illegitimate access by relying on the shared symmetric keys $skag$ and $skgs$ and the nons $T1$ and $T2$ to verify the requests sent between the different agents.

\subsection{Attack}
\label{sub:Attack}
In the protocol $PKINIT-short$ there is a man-in-the-middle-attack present in which the attacker $i$ obtains the initial authentication request between $C$ and $ath$ and subsequently authenticates on behalf of $C$, controlling all communication between $C$ and $g$.

This is possible since the initial request from $C$ to $ath$ is encrypted using $C$'s private key, which means everyone can read it since everyone has access to $C$'s public key. As such $i$ can easily decrypt the message and replace it's contents with it's own nons and private key encryption.

We can easily fix this, by encrypting the entire message send by $C$ using $ath$'s public key. This way $i$ will never be able decrypt the message, and subsequently fake the identity of $C$.
$C -> ath: \{C,g,N1,\{T0,N1,hash(C,g,N1)\}inv(pk(C))\}pk(ath)$

\subsection{Diffie-Hellman}
\label{sub:Diffie-Hellman}

In the following version of PKINIT we introduce the new shared prime $p$, and add $exp(p, X)$ to the initial request from $C$ to $ath$. $ath$ responds with $exp(p, Y)$ and encrypts $\{|g,KCG,T1,N1|\}$ with $exp(exp(p, Y), X))$ instead of $Ktemp$.\\

\begin{lstlisting}
Types: Agent C,ath,g,s;
   Number p,N1,N2,T0,T1,T2,Payload,tag;
   Function pk,hash,sk;
   Symmetric_key KCG,KCS,skag,skgs

Knowledge: C: C,ath,g,s,p,pk(ath),pk(C),inv(pk(C)),hash,tag,pk;
   ath: C,ath,g,p,pk(C),pk(ath),inv(pk(ath)),hash,skag,tag

where C!=ath

Actions:

C -> ath: {C,g,N1,{T0,N1,hash(C,g,N1)}inv(pk(C)),exp(p, X)}pk(ath)

ath -> C: C,
	({|ath,C,g,KCG,T1|}skag),
        (exp(p, Y)),
        ({|g,KCG,T1,N1|}exp(exp(p, Y), X))

Goals:
KCG secret between C,ath
\end{lstlisting}

\section{AMP}
\label{sec:AMP}

\subsection{Protocol description}
\label{sub:Protocol description}

$C$ knows about $s$ and $RP$ while $s$ only knows of $C$ and $RP$ only knows of $s$. They also know about each others public keys in the same manner.

The protocol tries to establish a relationship between C and RP, authenticating the two parties through $s$, enabling $RP$ to share $data$ with $C$.

The protocol tries to authenticate $C$ for $RP$, by letting $s$ validate the initial $Request$ and the $ReqID$ sent by $RP$.

\subsection{Attack analysis and fix}
\label{sub:Attack analysis and fix}

$i$ acts a man in the middle between $C$ and $RP$. $i$ receives a request from $C$, but forwards it to $RP$ imitating $C$. $i$ is able to authenticate on behalf of $C$ and retrieve $data$ which is a secret between $C$ and $RP$.

To fix the problem we sign the $Request$ with $C$'s private key. We can do this since $RP$ does not know $C$'s public key.
We also give $RP$'s public key to $s$, such that $s$'s can sign the message to $RP$ after successful authentication.

Now the $Request$ variable generated by $C$ is guaranteed to be unique throughout the protocol, preventing $i$ from spoofing $RP$.\\

\begin{lstlisting}
# We sign with C's private key, since RP does not know C's public key
C->RP: {C,RP,{Request}inv(pk(C)),K}pk(RP)
RP->C: {C,s,RP,ReqID,{Request}inv(pk(C))}K

# We give s RP's public key, such that s can encrypt its response, such that only RP can read it.
C->s:  { {C,s,pk(RP),ReqID,Request}inv(pk(C)) }pk(s)
s->C:  { {C,s,Request}inv(pk(s)) }pk(RP)

# Now Request is gauranteed to be unique for the whole request chain.
C->RP: {{C,s,Request}inv(pk(s))}pk(RP)
RP->C: {Data}K
\end{lstlisting}

\subsection{S as a dishonest server}
\label{sub:S as a dishonest server}

If the server is allowed to be dishonest, then we essentially do not need an authority for authentication. Any authority we might choose in the place of $S$, cannot be trusted to be honest, and thus we cannot use $S$ to validate our recipient.

There is simply no reason to use an untrusted third party to authenticate between to agents.

\end{document}
