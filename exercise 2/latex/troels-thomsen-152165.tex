\documentclass[12pt]{article}
\usepackage[a4paper, hmargin={2.5cm, 2.5cm}, vmargin={2.5cm, 2.5cm}]{geometry}

\usepackage[utf8]{inputenc}
\usepackage[english]{babel}
\usepackage{amssymb}
\usepackage{amsfonts}
\usepackage{amsmath}
\usepackage{setspace}
\usepackage{algorithm}
\usepackage[noend]{algpseudocode}

\usepackage{tikz}
\usetikzlibrary{positioning,shapes, shadows, arrows, automata}

\usepackage[T1]{fontenc}
\usepackage{inconsolata}

\usepackage{xcolor}
\usepackage{listings}
\usepackage{graphicx}
\usepackage[hidelinks]{hyperref}
\usepackage{float}
\usepackage[english]{varioref}
\usepackage{multirow}
\usepackage{hhline}
\usepackage{etoolbox}

\usepackage{fancyhdr}

\setlength\parindent{0pt}
\usepackage[parfill]{parskip}

\definecolor{pblue}{rgb}{0.13,0.13,1}
\definecolor{pgreen}{rgb}{0,0.5,0}
\definecolor{pred}{rgb}{0.9,0,0}
\definecolor{pgrey}{rgb}{0.46,0.45,0.48}
\definecolor{mygray}{rgb}{0.9451,0.9451,0.9451}

\lstset{
  backgroundcolor=\color{mygray},
  basicstyle=\footnotesize\ttfamily,
  mathescape,
  language=Java,
  commentstyle=\color{pgreen},
  keywordstyle=\color{pblue},
  stringstyle=\color{pred},
  breaklines=true,
  numbers=left,
  numberstyle=\ttfamily,
  stepnumber=1,
  firstnumber=1,
  numberfirstline=true,
  postbreak=\raisebox{0ex}[0ex][0ex]{\ensuremath{\color{red}\hookrightarrow\space}},
  literate={->}{$\rightarrow$}{2}
           {ε}{$\varepsilon$}{1}
}

\linespread{1.3}

\title{
  \vspace{4cm}
  \begin{flushleft}
  \Large{\textbf{Excercise 2 - Java Authentication}} \\
  \large{Data Security} \\
  \end{flushleft}
  \vspace{0cm}
  \begin{flushleft}
  \small
  \textit{\today}
  \end{flushleft}
  \vspace{12cm}
  \begin{flushleft}
  \small
  By Troels Thomsen \texttt{152165}\\
  Discussed with Rasmus Haarslev \texttt{152175}
  \end{flushleft}
}

\date{
	%
}

\begin{document}

\clearpage
\pagenumbering{gobble}
\thispagestyle{empty}
\maketitle

\newpage

\tableofcontents

\newpage

\pagenumbering{arabic}

\section{Introduction}
\label{sec:Introduction (1 page)}

The problem of authentication between server and client is one which is still being solve with many open solutions. The problem lies in the fact that the client needs to authenticate the server, and the server needs to authenticate the client in order to prevent man-in-the-middle attacks. For systems which require a user to sign in with a key, usually in the form of a password, the most common approach to solving the problem is adding an encryption layer on top of the key authentication.

This encryption layer, usually Transport Layer Security (TLS) or its predecessor Secure Sockets Layer (SSL), ensures privacy by encrypting data with a symmetric key which is agreed upon at the beginning of the sessions. This also ensures authenticity of the server, since the servers public key certificate is used to ensure its authenticity. In the ideal scenario the client will also share its public key certificate with the server, allowing the server to validate the authenticity of the client, however in the real world this is not practical since very few people have a personal private / public key pair. This is also the reason why most systemts have the added login with a username and password in the first place, since they cannot authenticate the users otherwise.

Given that anyone can establish a secure connection with the server puts a great deal of pressure on the servers ability to secure its stored user credentials, since they solely act a the sole method of user authentication and are practically the users private keys. Secure storage of passwords remains an industry-wide problem, with many services having exposed their users passwords to attackers in some way or another in recent years.

In the given scenario we need to authenticate a user with a printing service, allowing them to print files to networked printers controlled by the service. The proposed solution will try to solve the problem of authenticating users while securely storing their credentials. It will try to achieve this by storing the passwords in a hashed form using the SHA-512\footnote{\url{https://en.wikipedia.org/wiki/SHA-2}} hashing algorithm with a random 64-bit salt and 10000 hashing rounds.

The solution will assume that all users already exist in the system.
The solution does not try to solve the problem of securely transporting user credentials from the client to the server, and therefore expects the network to be secure.

\newpage

\section{Authentication}
\label{sec:Authentication}

\subsection{Password transport}
\label{sub:Password transport}

When relying on a password-based authentication scheme, the system must take into account the problem of securely transporting the users passwords. The system must work under the assumption that the users are connecting through an insecure network. Given this assumption it is vital that the system can utilize a secure protocol for engaging in encryption of the users authentication request. This means that before the users can authenticate with the system, they must agree upon some sort of encryption scheme.

When establishing a private session without any previously shared secret, we must first verify the origin of system with which the user is communicating. This is done by letting the server share a certificate which has been signed by one or more trusted third parties.
After the initial verification of the server the two parties must agree upon a cipher an a key. There are many ways the parties might go about this.
They could rely on a trusted third party to provide encryption keys given that both parties can securely communicate with this third party. This is a viable option but it does put a great deal of risk on the third party not being compromised, and it limits the possible users of the system to users who have knowledge of the trusted third party.

The parties could have a previously shared key, but we will not consider this case further since it is only useful in a limited set of scenarios.

The parties could also perform a variant of the Diffie–Hellman key exchange protocol, preferably one which authenticates the server and ensures forward secrecy like the Elliptic curve Diffie–Hellman protocol.\footnote{\url{https://en.wikipedia.org/wiki/Elliptic_curve_Diffie-Hellman}}

\newpage

\subsection{Password storage}
\label{sub:Password storage}

We will consider the following three storage scenarios.

\subsubsection{System password file}
\label{subs:System password file}

Storing the passwords in a file on the operating system has the added benefit of enabling the system to use the built-in permission and access system present in modern operating systems and file systems. This existing functionality could be utilized to confidentiality and integrity of the stored passwords.

The downside to the approach is the added development cost to the system. One would most likely have to write a significant amount of code to create efficient I/O utilization when searching and inserting.
Additionally this approach would be vulnerable to physical theft if only the existing access control functionality of the operating system was used without any encryption. If an attacker got hold of the harddrives, they would be able to gain access to all of the store passwords.

\subsubsection{Public password file}
\label{subs:Public password file}

Storing passwords in an encrypted file on the server is a slightly better solution that using a system password file. It avoids the downside of being vulnerable to physical theft since all the data would be encrypted, but still suffers from an increase in implementation time since one would most likely have to develop a tailored system for managing this authentication structure.

\subsubsection{Database storage}
\label{subs:Database storage}

The database system eliminates all of the downsides associated with the two other approaches. By using a well-established database provider, one can save the development cost incurred with the other solutions and still get an extremely efficient solution, which would most likely be better in every aspect than what could be tailored onto the system. By encrypting the private data stored in the database, the issue of physical or digital theft is also avoided.

The only downside is that the encryption / hashing algorithm implemented by the database might not be entirely up to date. This can usually be circumvented quite easily by implementing the hashing yourself.

\subsection{Password verification}
\label{sub:Password verification}

Verifying submitted passwords using a database requires the system to retrieve the record corresponding to the given username, and comparing the stored hashed password with a hash of the submitted password. It is here crucial that the submitted password is hashed with the same salt and number of hashing rounds as the one stored in the database.

If the two hashes are identical, we can consider the request to be valid.

\section{Design and implementation}
\label{sec:Design and implementation}

\subsection{Implementation choices}
\label{sub:Implementation choicesn}


The following implementation makes the assumption that the communication between the client and the server takes place on a secure network, and the implementation will therefore not try to solve the problem secure password transport.

Since the implementation is not going to try to solve the problem of secure password transport, it will instead focus on solving the problem of secure user authentication and password storage.

As a mechanism for storing user passwords we have chosen to use a SQL database, however we do not rely on the database's built-in password hashing. This is partly due to the fact that we have chosen a very simple setup with SQLite which do not have built-in functionality for hashing passwords like MySQL's \texttt{PASSWORD()} function. This has the added benefit of making the implementation portable between a virtually all SQL implementations with JDBC driver support. Another added benefit is that the developer of this implementation can make the choice of how they want the passwords to hashed since \texttt{java.security.MessageDigest} supports several cryptographic hashing functions, and the developer can further choose the salts and the number of hashing iterations used.

The reason we chose to use a database for password storage over a system file or a public password file is again portability. The solution can easily be ported to other systems since it does not rely on operating-system or file-system specific functionality. Using a database has the added benefit of relieving the developer from worrying about efficient file-system usage and similar issues which might arise if one were to try to implement a password file mechanism for storing passwords.

\subsection{Password hashing}
\label{sub:Password hashing}

The solution re-purposes the implementation of SHA-2 hashing with salts with the use of the \texttt{java.security} package found at \url{https://www.owasp.org/index.php/Hashing_Java}.

By default, unless a different salt is specified, the implementation generates a random 64-bit salt and uses it on a 512-bit SHA-2 digest. The clear text password is then hashed using the generated digest in 10000 iterations of re-hashing.
In order to be able to authenticate the user we need to store both the hash and the randomly generated salt.

We add the 64-bit salt in order to try to avoid having any to passwords hash to the same 512-bit value, and thus avoid attacks exploiting the birthday paradox.

By hashing over 10000 iterations we force any attacker to spend a factor of 10000 longer time when trying to break the hash. This adds to the time-cost for the system when generating hashes, but compared to the extra time added to the attackers procedure, the added iterations yields a good trade-off.

\begin{figure}[h!]
    \begin{lstlisting}
public Pair<String, String> hash(String password) throws UnsupportedEncodingException, NoSuchAlgorithmException {
    // Salt generation 64 bits long
    byte[] salt = new byte[8];
    new Random().nextBytes(salt);
    return hash(password, salt);
}

public Pair<String, String> hash(String password, byte[] salt) throws NoSuchAlgorithmException, UnsupportedEncodingException {
    MessageDigest digest = MessageDigest.getInstance("SHA-512");
    digest.reset();
    digest.update(salt);
    byte[] digestedPassword = digest.digest(password.getBytes("UTF-8"));

    for (int i = 0; i < ITERATION_NUMBER; i++) {
        digest.reset();
        digestedPassword = digest.digest(digestedPassword);
    }

    // Returns the hashed password and the salt
    return new Pair<String, String>(byteToBase64(digestedPassword), byteToBase64(salt));
}
    \end{lstlisting}
    \caption{Java code to generate the 512-bit SHA-2 hash.}
\end{figure}

\subsection{User authentication}
\label{sub:User authentication}

In order to authenticate the requests sent by the users, the service interface has been extended to include a user object containing a name and a password. This object is passed to a singleton called \texttt{UserLoginService} which handles the necessary database interaction and which utilizes the same hashing method as mentioned to validate the submitted user object.

This means that the solution considers each request as its own, and as such has no concept of a session. One could extend the current implementation and create a unique session hash on the server and store it in the user table, pass this session hash back to the client and let the client user it for some predefined number of seconds as a substitute for transmitting the actual user password. This would require some additional logic especially on part of the server, but would on the upside reduce the amount of times a user exposes their password to the network.

If the incoming user password hashes to the same value as is stored in the database, using the same random salt, we consider the user object valid.

\begin{figure}[h!]
    \begin{lstlisting}
Class.forName("org.sqlite.JDBC");
Connection c = DriverManager.getConnection("jdbc:sqlite:printservice.db");

ResultSet rs = c.createStatement().executeQuery(
        "SELECT password, salt FROM Users WHERE username = '" + user.getUsername() + "';"
);

if (rs.next()) {
    String digest = rs.getString("password");
    String salt = rs.getString("salt");

    // Compute the new digest
    Pair<String, String> hash = HashingService.getInstance().hash(
            user.getPassword(),
            HashingService.getInstance().base64ToByte(salt)
    );

    if (!hash.getKey().equals(digest)) {
        throw new LoginException();
    }
    return true;
}
    \end{lstlisting}
    \caption{Java code from \texttt{UserLoginService} to authenticate a user object.}
\end{figure}

\newpage

\section{Evaluation}
\label{sec:Evaluation}

In the solution there is a unit test called integration-test. Running this test will yield the following output:

\begin{lstlisting}[language=C]
We try to sign in with a valid account
0 : first.txt : printer1
1 : second.txt : printer1
2 : third.txt : printer2
We try to sign in with an invalid account
javax.security.auth.login.LoginException
	at server.UserLoginService.login(UserLoginService.java:49)
	at server.PrintServiceImpl.queue(PrintServiceImpl.java:44)
	at sun.reflect.NativeMethodAccessorImpl.invoke0(Native Method)
	at sun.reflect.NativeMethodAccessorImpl.invoke(NativeMethodAccessorImpl.java:62)
    ...
We could not sign in with invalid account.
\end{lstlisting}

The client tries to add three files to the printer queues with a valid and with an invalid user. As shown by the thrown \texttt{LoginException}, the invalid user is unable to successfully add anything to the queue.

\section{Conclusion}
\label{sec:Conclusion}

As stated the solution does not address the problem of securely transporting the user password from the client to the server. This could have been done and can be retrofitted onto the solution by utilizing the \texttt{SslRMIServerSocketFactory} interface or creating ones own \texttt{RMIServerSocketFactory} and specifying the desired protocol. This was never tried due to time constraints.

As demonstrated the solution successfully authenticates valid users for using the service, and likewise denies access to invalid users. It does this while storing the user passwords in a strongly hashed form, thus making their true value very unlikely to be found even in the case of the database being compromised.

%\newpage
%\appendix

%\section{Code}
%\label{sec:Code}

%\subsection{PrintService.java}
%\label{sub:PrintService.java}

%\lstinputlisting[label={appendix:service-interface}]{../RMI/src/main/java/shared/PrintService.java}

\end{document}
