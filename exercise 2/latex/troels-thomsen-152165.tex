\documentclass[12pt]{article}
\usepackage[a4paper, hmargin={2.5cm, 2.5cm}, vmargin={2.5cm, 2.5cm}]{geometry}

\usepackage[utf8]{inputenc}
\usepackage[english]{babel}
\usepackage{amssymb}
\usepackage{amsfonts}
\usepackage{amsmath}
\usepackage{setspace}
\usepackage{algorithm}
\usepackage[noend]{algpseudocode}

\usepackage{tikz}
\usetikzlibrary{positioning,shapes, shadows, arrows, automata}

\usepackage[T1]{fontenc}
\usepackage{inconsolata}

\usepackage{xcolor}
\usepackage{listings}
\usepackage{graphicx}
\usepackage[hidelinks]{hyperref}
\usepackage{float}
\usepackage[english]{varioref}
\usepackage{multirow}
\usepackage{hhline}
\usepackage{etoolbox}

\usepackage{fancyhdr}

\setlength\parindent{0pt}
\usepackage[parfill]{parskip}

\definecolor{pblue}{rgb}{0.13,0.13,1}
\definecolor{pgreen}{rgb}{0,0.5,0}
\definecolor{pred}{rgb}{0.9,0,0}
\definecolor{pgrey}{rgb}{0.46,0.45,0.48}
\definecolor{mygray}{rgb}{0.9451,0.9451,0.9451}

\lstset{
  backgroundcolor=\color{mygray},
  basicstyle=\footnotesize\ttfamily,
  mathescape,
  language=Java,
  commentstyle=\color{pgreen},
  keywordstyle=\color{pblue},
  stringstyle=\color{pred},
  breaklines=true,
  numbers=left,
  numberstyle=\ttfamily,
  stepnumber=1,
  firstnumber=1,
  numberfirstline=true,
  postbreak=\raisebox{0ex}[0ex][0ex]{\ensuremath{\color{red}\hookrightarrow\space}},
  literate={->}{$\rightarrow$}{2}
           {ε}{$\varepsilon$}{1}
}

\linespread{1.3}

\title{
  \vspace{4cm}
  \begin{flushleft}
  \Large{\textbf{Excercise 2 - Java Authentication}} \\
  \large{Data Security} \\
  \end{flushleft}
  \vspace{0cm}
  \begin{flushleft}
  \small
  \textit{\today}
  \end{flushleft}
  \vspace{12cm}
  \begin{flushleft}
  \small
  Troels Thomsen \texttt{152165}
  \end{flushleft}
}

\date{
	%
}

\begin{document}

\clearpage
\pagenumbering{gobble}
\thispagestyle{empty}
\maketitle

\newpage

\pagenumbering{arabic}

\section{Introduction}
\label{sec:Introduction}

\newpage

\section{Authentication}
\label{sec:Authentication}

\newpage

\section{Design and implementation}
\label{sec:Design and implementation}

My implementation is based on the following understanding of the problem.

\begin{enumerate}
    \item Implement a RMI server, exposing the interface shown in Figure~\ref{fig:service-interface} to the client.
    \item Authenticate users on the server before allowing them to use the RMI service.
    \begin{itemize}
        \item The server assumes users already exist and do not handle creation of new users.
        \item The implementation of this authentication should include a secure user password storage.
    \end{itemize}
\end{enumerate}

\begin{figure}[h!]
    \begin{lstlisting}
    package service;

    import service.model.PrintJobQueue;
    import java.rmi.Remote;

    public interface PrintService extends Remote {

        void print(String filename, String printer);
        PrintJobQueue queue();
        void topQueue(int jobId);
        void start();
        void stop();
        void restart();
        String status();
        String readConfig(String parameter);
        void setConfig(String parameter, String value);
    }
    \end{lstlisting}
    \label{fig:service-interface}
    \caption{The interface exposed through RMI.}
\end{figure}

\newpage

\section{Evaluation}
\label{sec:Evaluation}

\newpage

\section{Conclusion}
\label{sec:Conclusion}


\end{document}
